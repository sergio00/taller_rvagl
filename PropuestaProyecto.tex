% !TEX TS-program = pdflatex
% !TEX encoding = UTF-8 Unicode

% This is a simple template for a LaTeX document using the "article" class.
% See "book", "report", "letter" for other types of document.

\documentclass[11pt]{article} % use larger type; default would be 10pt

\usepackage[utf8]{inputenc} % set input encoding (not needed with XeLaTeX)
\usepackage[spanish]{babel}

%%% Examples of Article customizations
% These packages are optional, depending whether you want the features they provide.
% See the LaTeX Companion or other references for full information.

%%% PAGE DIMENSIONS
\usepackage{geometry} % to change the page dimensions
\geometry{a4paper} % or letterpaper (US) or a5paper or....
% \geometry{margin=2in} % for example, change the margins to 2 inches all round
% \geometry{landscape} % set up the page for landscape
%   read geometry.pdf for detailed page layout information

\usepackage{graphicx} % support the \includegraphics command and options

% \usepackage[parfill]{parskip} % Activate to begin paragraphs with an empty line rather than an indent

%%% PACKAGES
\usepackage{booktabs} % for much better looking tables
\usepackage{array} % for better arrays (eg matrices) in maths
%\usepackage{paralist} % very flexible & customisable lists (eg. enumerate/itemize, etc.)
\usepackage{verbatim} % adds environment for commenting out blocks of text & for better verbatim
\usepackage{subfig} % make it possible to include more than one captioned figure/table in a single float
% These packages are all incorporated in the memoir class to one degree or another...

%%% HEADERS & FOOTERS
\usepackage{fancyhdr} % This should be set AFTER setting up the page geometry
\pagestyle{fancy} % options: empty , plain , fancy
\renewcommand{\headrulewidth}{0pt} % customise the layout...
\lhead{}\chead{}\rhead{}
\lfoot{}\cfoot{\thepage}\rfoot{}

%%% SECTION TITLE APPEARANCE
\usepackage{sectsty}
\allsectionsfont{\sffamily\mdseries\upshape} % (See the fntguide.pdf for font help)
% (This matches ConTeXt defaults)

%%% ToC (table of contents) APPEARANCE
\usepackage[nottoc,notlof,notlot]{tocbibind} % Put the bibliography in the ToC
\usepackage[titles,subfigure]{tocloft} % Alter the style of the Table of Contents
\renewcommand{\cftsecfont}{\rmfamily\mdseries\upshape}
\renewcommand{\cftsecpagefont}{\rmfamily\mdseries\upshape} % No bold!

%%% END Article customizations

%%% The "real" document content comes below...

\title{Pizarra en 3 Dimensiones}
\author{David Barrera\\Xavier Moncayo}

\begin{document}
\maketitle
\tableofcontents
\section{Introduccion y Justificacion}
Actualmente en la educacion es dificil encontrar una manera de explicar conceptos que requieren una perspectiva en 3 dimensiones, desde algebra lineal, calculo de varias variables, calculo integral, fisica, muchas materias requieren graficos, que en una pizarra en 2 dimensiones es dificil encontrarle sentido.
El impacto que puede tener el proyecto esta orientado basicamente a la educacion, pero bien puede servir como base para otro tipo de actividades como modelamiento en 3D.
\section{Marco Teorico}
Actualmente se busca siempre tener un ritmo acelerado para aprender, sobre todo cuestiones fisicas y espaciales, que serviran de base en materias posteriores y que idealmente debe de ser aprendido de ser posible hasta instintivamente, una buena manera de hacerlo es mostrando a los alumnos graficos en una pizarra que ilustren nociones matematicas, lo cual, es bastante util cuando se trata en 2 dimensiones, pero cuando se llegan a 3 dimensiones, la dificultad de entenderlo rapidamente se complica.
Han habido otros proyectos que han intentado hacer algo similar (o igual)
"We present a method for tracking the 3D position of a finger, using a single camera placed several meters away from the user. After skin detection, we use motion to identify the gesticulating arm. The finger point is found by analyzing the arm's outline. To derive a 3D trajectory, we first track 2D positions of the user's elbow and shoulder."[1]
\section{Recursos}
Los recursos con los cuales contamos, son equipos del CTI, proyectores, camaras infrarrojas y un  casco neuronal

[1] Leido en la pagina: http://citeseerx.ist.psu.edu/viewdoc/summary?doi=10.1.1.40.5510
http://vision.eecs.ucf.edu/projects/BlackBoard/fg2000.pdf
\end{document}
